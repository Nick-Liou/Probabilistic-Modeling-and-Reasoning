\documentclass{article}

% Import packages from local file
\usepackage{packages}

% Define the homework number as a variable
\newcommand{\HomeworkNumber}{4}

\begin{document}

% Add the cover


% Set up custom date format
\newdateformat{monthyeardate}{\monthname[\THEMONTH] \THEYEAR}

% Header and Footer configuration
\pagestyle{fancy}
\fancyhf{}
\fancyhead[L]{\leftmark}
\fancyhead[R]{\thepage}
\fancyfoot[C]{Probabilistic Modeling and Reasoning Homework \HomeworkNumber}

% Chapter and Section formatting
\titleformat{\chapter}[display]
  {\normalfont\bfseries}{}{0pt}{\Huge}
\titlespacing*{\chapter}{0pt}{-20pt}{20pt}

% Adjust header height and top margin
\setlength{\headheight}{14.49998pt}
\addtolength{\topmargin}{-2.49998pt}

% % Custom abstract environment
% \newenvironment{customabstract}
%   {\vspace*{1cm}
%    \begin{center}
%    \bfseries \huge Abstract
%    \end{center}
%    \vspace{0.5cm}
%    \normalfont \large}
%   {\vspace{1cm}}



\makeatletter
% Taken from http://ctan.org/pkg/centernot
\newcommand*{\centernot}{%
  \mathpalette\@centernot
}
\def\@centernot#1#2{%
  \mathrel{%
    \rlap{%
      \settowidth\dimen@{$\m@th#1{#2}$}%
      \kern.5\dimen@
      \settowidth\dimen@{$\m@th#1=$}%
      \kern-.5\dimen@
      $\m@th#1\not$%
    }%
    {#2}%
  }%
}
\makeatother


\begin{titlepage}
    \centering

    % University logo
    \vfill
    \includegraphics[width=0.3\textwidth]{UOMLOGOEN.eps}
    \vfill

    % Main title
    {\Huge \textbf{Probabilistic Modeling and Reasoning}} \\
    {\LARGE Homework — \HomeworkNumber}

    \vfill  % Vertical fill for dynamic centering
    % Authors' names
    {\Large \textbf{Nikolaos Liouliakis (AID25001)}} \\
    {\Large \textbf{Vasileios-Efraim Tsavalia (AID25006)}}

    % \vfill

    % % Course submission information
    % {\Large A report submitted for the course} \\
    % {\Large \textbf{Probabilistic Modeling and Reasoning}}

    \vfill

    % Program and University information
    {\Large MSc in Artificial Intelligence and Data Analytics} \\
    {\Large University of Macedonia}

    \vfill

    % Supervisor information
    {\Large \textbf{Supervisor: Professor Dimitris Christou-Varsakelis}}

    \vfill

    % Date with custom format
    {\Large \monthyeardate\today} % Automatically displays in "October 2024" format
    
\end{titlepage}


\newpage



\section*{Problem 1}
\renewcommand{\arraystretch}{1.5}

\subsection*{Part 1}

When learning the conditional probability tables (CPTs) using maximum likelihood, we will prove in Problem 3 that it yields the empirical distribution. Therefore, after renaming the variables the CPT are:


\begin{table}[H]
\centering
\begin{tabular}{|c|c|}
\hline
\textbf{Old variable name} & \textbf{New variable name} \\ \hline
fuse assembly malfunction & $x_0$ \\ \hline
drum unit                 & $x_1$ \\ \hline
toner out                 & $x_2$ \\ \hline
poor paper quality        & $x_3$ \\ \hline
worn roller               & $x_4$ \\ \hline
burning smell             & $x_5$ \\ \hline
poor print quality        & $x_6$ \\ \hline
wrinkled pages            & $x_7$ \\ \hline
multiple pages fed        & $x_8$ \\ \hline
paper jam                 & $x_9$ \\ \hline
\end{tabular}
\caption{Mapping of original variable names to new variable names.}
\label{tab:my-table}
\end{table}


% fuse assembly malfunction
% drum unit
% toner out
% poor paper quality
% worn roller
% burning smell
% poor print quality
% wrinkled pages
% multiple pages fed
% paper jam

\begin{table}[H]
\centering
\begin{tabular}{|c|}
\hline
$P(X_0 = 1)$  \\ \hline
$\frac{3}{15}$ \\ \hline
\end{tabular}
\caption{CPT for $X_0$}
\end{table}

\begin{table}[H]
\centering
\begin{tabular}{|c|}
\hline
$P(X_1 = 1)$  \\ \hline
$\frac{4}{15}$ \\ \hline
\end{tabular}
\caption{CPT for $X_1$}
\end{table}

\begin{table}[H]
\centering
\begin{tabular}{|c|}
\hline
$P(X_2 = 1)$  \\ \hline
$\frac{5}{15}$  \\ \hline
\end{tabular}
\caption{CPT for $X_2$}
\end{table}

\begin{table}[H]
\centering
\begin{tabular}{|c|}
\hline
$P(X_3 = 1)$  \\ \hline
$\frac{8}{15}$ \\ \hline
\end{tabular}
\caption{CPT for $X_3$}
\end{table}

\begin{table}[H]
\centering
\begin{tabular}{|c|}
\hline
$P(X_4 = 1)$  \\ \hline
$\frac{3}{15}$  \\ \hline
\end{tabular}
\caption{CPT for $X_4$}
\end{table}

\begin{table}[H]
\centering
\begin{tabular}{|c|c|}
\hline
$P(X_5 = 1 | X_0)$ & $X_0$ \\ \hline
$\frac{0}{12}$ & 0 \\ \hline
$\frac{2}{3}$ & 1 \\ \hline
\end{tabular}
\caption{CPT for $X_5$}
\end{table}

\begin{table}[H]
\centering
\begin{tabular}{|c|c|c|c|c|}
\hline
$P(X_6 = 1 | X_1, X_2, X_3)$ & $X_1$ & $X_2$ & $X_3$ \\ \hline
$\frac{0}{3}$ & 0 & 0 & 0 \\ \hline
$\frac{1}{5}$ & 0 & 0 & 1 \\ \hline
$\frac{2}{2}$ & 0 & 1 & 0 \\ \hline
$\frac{1}{1}$ & 0 & 1 & 1 \\ \hline
$\frac{1}{1}$ & 1 & 0 & 0 \\ \hline
$\frac{1}{1}$ & 1 & 0 & 1 \\ \hline
$\frac{1}{1}$ & 1 & 1 & 0 \\ \hline
$\frac{1}{1}$ & 1 & 1 & 1 \\ \hline
\end{tabular}
\caption{CPT for $X_6$}
\end{table}

\begin{table}[H]
\centering
\begin{tabular}{|c|c|c|}
\hline
$P(X_7 = 1 | X_0, X_3)$ & $X_0$ & $X_3$ \\ \hline
$\frac{1}{5}$ & 0 & 0 \\ \hline
$\frac{2}{7}$ & 0 & 1 \\ \hline
$\frac{1}{2}$ & 1 & 0 \\ \hline
$\frac{1}{1}$ & 1 & 1 \\ \hline
\end{tabular}
\caption{CPT for $X_7$}
\end{table}

\begin{table}[H]
\centering
\begin{tabular}{|c|c|c|}
\hline
$P(X_8 = 1 | X_3, X_4)$ & $X_3$ & $X_4$ \\ \hline
$\frac{0}{5}$ & 0 & 0 \\ \hline
$\frac{1}{2}$ & 0 & 1 \\ \hline
$\frac{2}{7}$ & 1 & 0 \\ \hline
$\frac{1}{1}$ & 1 & 1 \\ \hline
\end{tabular}
\caption{CPT for $X_8$}
\end{table}

\begin{table}[H]
\centering
\begin{tabular}{|c|c|c|}
\hline
$P(X_9 = 1 | X_0, X_4)$ & $X_0$ & $X_4$ \\ \hline
$\frac{4}{10}$ & 0 & 0 \\ \hline
$\frac{2}{2}$ & 0 & 1 \\ \hline
$\frac{1}{2}$ & 1 & 0 \\ \hline
$\frac{1}{1}$ & 1 & 1 \\ \hline
\end{tabular}
\caption{CPT for $X_9$}
\end{table}


The CPT tables were calculated using the "printer.mat" file and the "CPT\_helper.m" MATLAB script. 


\subsection*{Part 2}

The probability of a fuse assembly malfunction, given that the secretary reports a burning smell, a paper jam, and no other issues, is:

\[
P(X_0 = 1 | X_5 = 1, X_6 = 0, X_7 = 0, X_8 = 0, X_9 = 1) = 1.0
\]

To calculate this probability, the CPTs were stored in the "printer.txt" file and loaded into the python script "BN\_prob\_calculator.py"


\subsection*{Part 3}

To calculate the CPTs using a Bayesian method with a uniform Beta prior $B(1,1)$, we adjust the formula for parameter estimation by adding $a=1$ to the numerator and adding $a+b=2$ to the denominator.


The new CPTs are:


\begin{table}[H]
\centering
\begin{tabular}{|c|}
\hline
$P(X_0 = 1)$  \\ \hline
$\frac{4}{17}$ \\ \hline
\end{tabular}
\caption{CPT for $X_0$}
\end{table}

\begin{table}[H]
\centering
\begin{tabular}{|c|}
\hline
$P(X_1 = 1)$  \\ \hline
$\frac{5}{17}$ \\ \hline
\end{tabular}
\caption{CPT for $X_1$}
\end{table}

\begin{table}[H]
\centering
\begin{tabular}{|c|}
\hline
$P(X_2 = 1)$  \\ \hline
$\frac{6}{17}$  \\ \hline
\end{tabular}
\caption{CPT for $X_2$}
\end{table}

\begin{table}[H]
\centering
\begin{tabular}{|c|}
\hline
$P(X_3 = 1)$  \\ \hline
$\frac{9}{17}$ \\ \hline
\end{tabular}
\caption{CPT for $X_3$}
\end{table}

\begin{table}[H]
\centering
\begin{tabular}{|c|}
\hline
$P(X_4 = 1)$  \\ \hline
$\frac{4}{17}$  \\ \hline
\end{tabular}
\caption{CPT for $X_4$}
\end{table}

\begin{table}[H]
\centering
\begin{tabular}{|c|c|}
\hline
$P(X_5 = 1 | X_0)$ & $X_0$ \\ \hline
$\frac{1}{14}$ & 0 \\ \hline
$\frac{3}{5}$ & 1 \\ \hline
\end{tabular}
\caption{CPT for $X_5$}
\end{table}

\begin{table}[H]
\centering
\begin{tabular}{|c|c|c|c|c|}
\hline
$P(X_6 = 1 | X_1, X_2, X_3)$ & $X_1$ & $X_2$ & $X_3$ \\ \hline
$\frac{1}{5}$ & 0 & 0 & 0 \\ \hline
$\frac{2}{7}$ & 0 & 0 & 1 \\ \hline
$\frac{3}{4}$ & 0 & 1 & 0 \\ \hline
$\frac{2}{3}$ & 0 & 1 & 1 \\ \hline
$\frac{2}{3}$ & 1 & 0 & 0 \\ \hline
$\frac{2}{3}$ & 1 & 0 & 1 \\ \hline
$\frac{2}{3}$ & 1 & 1 & 0 \\ \hline
$\frac{2}{3}$ & 1 & 1 & 1 \\ \hline
\end{tabular}
\caption{CPT for $X_6$}
\end{table}

\begin{table}[H]
\centering
\begin{tabular}{|c|c|c|}
\hline
$P(X_7 = 1 | X_0, X_3)$ & $X_0$ & $X_3$ \\ \hline
$\frac{2}{7}$ & 0 & 0 \\ \hline
$\frac{3}{9}$ & 0 & 1 \\ \hline
$\frac{2}{4}$ & 1 & 0 \\ \hline
$\frac{2}{3}$ & 1 & 1 \\ \hline
\end{tabular}
\caption{CPT for $X_7$}
\end{table}

\begin{table}[H]
\centering
\begin{tabular}{|c|c|c|}
\hline
$P(X_8 = 1 | X_3, X_4)$ & $X_3$ & $X_4$ \\ \hline
$\frac{1}{7}$ & 0 & 0 \\ \hline
$\frac{2}{4}$ & 0 & 1 \\ \hline
$\frac{3}{9}$ & 1 & 0 \\ \hline
$\frac{2}{3}$ & 1 & 1 \\ \hline
\end{tabular}
\caption{CPT for $X_8$}
\end{table}

\begin{table}[H]
\centering
\begin{tabular}{|c|c|c|}
\hline
$P(X_9 = 1 | X_0, X_4)$ & $X_0$ & $X_4$ \\ \hline
$\frac{5}{12}$ & 0 & 0 \\ \hline
$\frac{3}{4}$ & 0 & 1 \\ \hline
$\frac{2}{4}$ & 1 & 0 \\ \hline
$\frac{2}{3}$ & 1 & 1 \\ \hline
\end{tabular}
\caption{CPT for $X_9$}
\end{table}


The probability of a fuse assembly malfunction, given that the secretary reports a burning smell, a paper jam, and no other issues, is:

\[
P(X_0 = 1 | X_5 = 1, X_6 = 0, X_7 = 0, X_8 = 0, X_9 = 1) = 0.64123752
\]


To calculate this probability, the CPT tables were stored in the "printer\_beta\_prior.txt" file and loaded into the python script "BN\_prob\_calculator.py"


\subsection*{Part 4}


\begin{table}[H]
\centering
\begin{tabular}{|l|l|}
\hline
$P(X_0 = 0, X_1 = 0, X_2 = 0, X_3 = 0, X_4 = 0 | X_5 = 1, X_6 = 0, X_7 = 0, X_8 = 0, X_9 = 1)$                      & 0.10704581                      \\ \hline
$P(X_0 = 0, X_1 = 0, X_2 = 0, X_3 = 0, X_4 = 1 | X_5 = 1, X_6 = 0, X_7 = 0, X_8 = 0, X_9 = 1)$                      & 0.03458403                      \\ \hline
$P(X_0 = 0, X_1 = 0, X_2 = 0, X_3 = 1, X_4 = 0 | X_5 = 1, X_6 = 0, X_7 = 0, X_8 = 0, X_9 = 1)$                      & 0.07805424                      \\ \hline
$P(X_0 = 0, X_1 = 0, X_2 = 0, X_3 = 1, X_4 = 1 | X_5 = 1, X_6 = 0, X_7 = 0, X_8 = 0, X_9 = 1)$                      & 0.02161502                      \\ \hline
$P(X_0 = 0, X_1 = 0, X_2 = 1, X_3 = 0, X_4 = 0 | X_5 = 1, X_6 = 0, X_7 = 0, X_8 = 0, X_9 = 1)$                      & 0.01824644                      \\ \hline
$P(X_0 = 0, X_1 = 0, X_2 = 1, X_3 = 0, X_4 = 1 | X_5 = 1, X_6 = 0, X_7 = 0, X_8 = 0, X_9 = 1)$                      & 0.00589500                      \\ \hline
$P(X_0 = 0, X_1 = 0, X_2 = 1, X_3 = 1, X_4 = 0 | X_5 = 1, X_6 = 0, X_7 = 0, X_8 = 0, X_9 = 1)$                      & 0.01986835                      \\ \hline
$P(X_0 = 0, X_1 = 0, X_2 = 1, X_3 = 1, X_4 = 1 | X_5 = 1, X_6 = 0, X_7 = 0, X_8 = 0, X_9 = 1)$                      & 0.00550200                      \\ \hline
$P(X_0 = 0, X_1 = 1, X_2 = 0, X_3 = 0, X_4 = 0 | X_5 = 1, X_6 = 0, X_7 = 0, X_8 = 0, X_9 = 1)$                      & 0.01858434                      \\ \hline
$P(X_0 = 0, X_1 = 1, X_2 = 0, X_3 = 0, X_4 = 1 | X_5 = 1, X_6 = 0, X_7 = 0, X_8 = 0, X_9 = 1)$                      & 0.00600417                      \\ \hline
$P(X_0 = 0, X_1 = 1, X_2 = 0, X_3 = 1, X_4 = 0 | X_5 = 1, X_6 = 0, X_7 = 0, X_8 = 0, X_9 = 1)$                      & 0.01517721                      \\ \hline
$P(X_0 = 0, X_1 = 1, X_2 = 0, X_3 = 1, X_4 = 1 | X_5 = 1, X_6 = 0, X_7 = 0, X_8 = 0, X_9 = 1)$                      & 0.00420292                      \\ \hline
$P(X_0 = 0, X_1 = 1, X_2 = 1, X_3 = 0, X_4 = 0 | X_5 = 1, X_6 = 0, X_7 = 0, X_8 = 0, X_9 = 1)$                      & 0.01013691                      \\ \hline
$P(X_0 = 0, X_1 = 1, X_2 = 1, X_3 = 0, X_4 = 1 | X_5 = 1, X_6 = 0, X_7 = 0, X_8 = 0, X_9 = 1)$                      & 0.00327500                      \\ \hline
$P(X_0 = 0, X_1 = 1, X_2 = 1, X_3 = 1, X_4 = 0 | X_5 = 1, X_6 = 0, X_7 = 0, X_8 = 0, X_9 = 1)$                      & 0.00827847                      \\ \hline
$P(X_0 = 0, X_1 = 1, X_2 = 1, X_3 = 1, X_4 = 1 | X_5 = 1, X_6 = 0, X_7 = 0, X_8 = 0, X_9 = 1)$                      & 0.00229250                      \\ \hline
$P(X_0 = 1, X_1 = 0, X_2 = 0, X_3 = 0, X_4 = 0 | X_5 = 1, X_6 = 0, X_7 = 0, X_8 = 0, X_9 = 1)$                      & 0.23240468                      \\ \hline
$P(X_0 = 1, X_1 = 0, X_2 = 0, X_3 = 0, X_4 = 1 | X_5 = 1, X_6 = 0, X_7 = 0, X_8 = 0, X_9 = 1)$                      & 0.05561821                      \\ \hline
$P(X_0 = 1, X_1 = 0, X_2 = 0, X_3 = 1, X_4 = 0 | X_5 = 1, X_6 = 0, X_7 = 0, X_8 = 0, X_9 = 1)$                      & 0.12104412                      \\ \hline
$P(X_0 = 1, X_1 = 0, X_2 = 0, X_3 = 1, X_4 = 1 | X_5 = 1, X_6 = 0, X_7 = 0, X_8 = 0, X_9 = 1)$                      & 0.02482956                      \\ \hline
$P(X_0 = 1, X_1 = 0, X_2 = 1, X_3 = 0, X_4 = 0 | X_5 = 1, X_6 = 0, X_7 = 0, X_8 = 0, X_9 = 1)$                      & 0.03961443                      \\ \hline
$P(X_0 = 1, X_1 = 0, X_2 = 1, X_3 = 0, X_4 = 1 | X_5 = 1, X_6 = 0, X_7 = 0, X_8 = 0, X_9 = 1)$                      & 0.00948037                      \\ \hline
$P(X_0 = 1, X_1 = 0, X_2 = 1, X_3 = 1, X_4 = 0 | X_5 = 1, X_6 = 0, X_7 = 0, X_8 = 0, X_9 = 1)$                      & 0.03081122                      \\ \hline
$P(X_0 = 1, X_1 = 0, X_2 = 1, X_3 = 1, X_4 = 1 | X_5 = 1, X_6 = 0, X_7 = 0, X_8 = 0, X_9 = 1)$                      & 0.00632025                      \\ \hline
$P(X_0 = 1, X_1 = 1, X_2 = 0, X_3 = 0, X_4 = 0 | X_5 = 1, X_6 = 0, X_7 = 0, X_8 = 0, X_9 = 1)$                      & 0.04034803                      \\ \hline
$P(X_0 = 1, X_1 = 1, X_2 = 0, X_3 = 0, X_4 = 1 | X_5 = 1, X_6 = 0, X_7 = 0, X_8 = 0, X_9 = 1)$                      & 0.00965593                      \\ \hline
$P(X_0 = 1, X_1 = 1, X_2 = 0, X_3 = 1, X_4 = 0 | X_5 = 1, X_6 = 0, X_7 = 0, X_8 = 0, X_9 = 1)$                      & 0.02353635                      \\ \hline
$P(X_0 = 1, X_1 = 1, X_2 = 0, X_3 = 1, X_4 = 1 | X_5 = 1, X_6 = 0, X_7 = 0, X_8 = 0, X_9 = 1)$                      & 0.00482797                      \\ \hline
$P(X_0 = 1, X_1 = 1, X_2 = 1, X_3 = 0, X_4 = 0 | X_5 = 1, X_6 = 0, X_7 = 0, X_8 = 0, X_9 = 1)$                      & 0.02200801                      \\ \hline
$P(X_0 = 1, X_1 = 1, X_2 = 1, X_3 = 0, X_4 = 1 | X_5 = 1, X_6 = 0, X_7 = 0, X_8 = 0, X_9 = 1)$                      & 0.00526687                      \\ \hline
$P(X_0 = 1, X_1 = 1, X_2 = 1, X_3 = 1, X_4 = 0 | X_5 = 1, X_6 = 0, X_7 = 0, X_8 = 0, X_9 = 1)$                      & 0.01283801                      \\ \hline
$P(X_0 = 1, X_1 = 1, X_2 = 1, X_3 = 1, X_4 = 1 | X_5 = 1, X_6 = 0, X_7 = 0, X_8 = 0, X_9 = 1)$                      & 0.00263343                      \\ \hline
\end{tabular}
\caption{All possible diagnosis given the evidence}
\end{table}

Therefore, the most likely diagnosis is: 

\[
P(X_0 = 1, X_1 = 0, X_2 = 0, X_3 = 0, X_4 = 0 | X_5 = 1, X_6 = 0, X_7 = 0, X_8 = 0, X_9 = 1) = 0.232404
\]

To calculate the above probabilities, the CPTs were stored in the "printer\_beta\_prior.txt" file and loaded into the python script "BN\_prob\_calculator.py"



\section*{Problem 2}

\begin{enumerate}
    \item \textbf{Fitting a Gaussian to each class:}
    
    Let the training examples from class 1 be:
    \[
    x_1 = \{0.5, 0.1, 0.2, 0.4, 0.3, 0.2, 0.2, 0.1, 0.35, 0.25\}
    \]
    and from class 2:
    \[
    x_2 = \{0.9, 0.8, 0.75, 1.0\}
    \]

    To fit a Gaussian distribution, we need to estimate the mean and variance for each class using Maximum Likelihood Estimation (MLE).

    \begin{itemize}
        \item For class 1, the mean $\mu_1$ and variance $\sigma_1^2$ are given by:
        \[
        \mu_1 = \frac{1}{n_1} \sum_{i=1}^{n_1} x_{1,i} = 0.26, \quad \sigma_1^2 = \frac{1}{n_1} \sum_{i=1}^{n_1} (x_{1,i} - \mu_1)^2 = 0.0149
        \]
        where $n_1 = 10$.

        \item For class 2, the mean $\mu_2$ and variance $\sigma_2^2$ are:
        \[
        \mu_2 = \frac{1}{n_2} \sum_{i=1}^{n_2} x_{2,i} = 0.8625, \quad \sigma_2^2 = \frac{1}{n_2} \sum_{i=1}^{n_2} (x_{2,i} - \mu_2)^2 = 0.0092
        \]
        where $n_2 = 4$.
    \end{itemize}

    \item \textbf{Estimating class probabilities:}
    
    The class probabilities \( p_1 \) and \( p_2 \) can be estimated by the relative frequencies of each class in the dataset:
    \[
    p_1 = \frac{n_1}{n_1 + n_2} = \frac{10}{14}, \quad p_2 = \frac{n_2}{n_1 + n_2}  = \frac{4}{14}
    \]

    \item \textbf{Calculating the probability that \( x = 0.6 \) belongs to class 1:}
    
    Using Bayes' theorem, we have:
    \[
    P(\text{class } 1 | x = 0.6) = \frac{P(x = 0.6 | \text{class } 1) \cdot p_1}{P(x = 0.6 | \text{class } 1) \cdot p_1 + P(x = 0.6 | \text{class } 2) \cdot p_2}
    \]
    where
    \begin{equation*}
    \begin{aligned}        
        P(x = 0.6 | \text{class } 1) 
        &= \frac{1}{\sqrt{2 \pi \sigma_1^2}} \exp\left(-\frac{(0.6 - \mu_1)^2}{2 \sigma_1^2}\right) \\
        &= \frac{1}{\sqrt{2 \pi \cdot 0.0149}} \exp\left(-\frac{(0.6 - 0.26)^2}{2 \cdot 0.0149}\right)  \\
        &= \frac{1}{\sqrt{2 \pi }} \frac{1}{0.1221} 0.0206675   \\ 
        &= \frac{1}{\sqrt{2 \pi }}  0.1692669942669
    \end{aligned}
    \end{equation*}
    
    
    and
    \[
    \begin{aligned}        
    P(x = 0.6 | \text{class } 2) 
    &= \frac{1}{\sqrt{2 \pi \sigma_2^2}} \exp\left(-\frac{(0.6 - \mu_2)^2}{2 \sigma_2^2}\right) \\
    &= \frac{1}{\sqrt{2 \pi \cdot 0.0092}} \exp\left(-\frac{(0.6 - 0.8625)^2}{2 \cdot 0.0092}\right) \\    
    &= \frac{1}{\sqrt{2 \pi }} \frac{1}{0.0960} 0.0236379   \\ 
    &= \frac{1}{\sqrt{2 \pi }} 0.246228125
    \end{aligned}
    \]

    Substituting we get:
    
     \[
    P(\text{class } 1 | x = 0.6) = \frac{ \frac{1}{\sqrt{2 \pi }}  0.1692669942669  \cdot \frac{10}{14}}{\frac{1}{\sqrt{2 \pi }}  0.1692669942669 \cdot \frac{10}{14} + \frac{1}{\sqrt{2 \pi }} 0.246228125 \cdot \frac{4}{14}} 
    = 0.63216
    \]

\end{enumerate}



\section*{Problem 3}



We need to show that the maximum likelihood estimate for the parameters \( \theta_s^i(t_i) = p(x_i = s \mid \operatorname{pa}(x_i) = t_i) \) is given by:

\[
\theta_s^i(t_i) = \frac{\sum_{n=1}^N \mathbb{I}\left[x_i^n = s\right] \mathbb{I}\left[\operatorname{pa}(x_i^n) = t_i\right]}{\sum_{n=1}^N \sum_s \mathbb{I}\left[x_i^n = s\right] \mathbb{I}\left[\operatorname{pa}(x_i^n) = t_i\right]}
\]

\textbf{Define the Log Likelihood of the Data}

The log likelihood for the dataset \( \mathcal{X} = \{ \mathbf{x}^1, \dots, \mathbf{x}^N \} \), assuming the belief network structure and independently gathered observations, is given by:
\[
\log p(\mathcal{X}) = \sum_{n=1}^N \sum_{i=1}^K \log p\left(x_i^n \mid \operatorname{pa}(x_i^n)\right)
\]

% ### Step 2: Define \( \theta_s^i(t_i) \)

The parameter \( \theta_s^i(t_i) = p(x_i = s \mid \operatorname{pa}(x_i) = t_i) \), represents the probability that variable \( x_i \) is in state \( s \) given that its parent variables are in the state \( t_i \).

\textbf{Formulate the Lagrangian}

To find the maximum likelihood estimate of \( \theta_s^i(t_i) \), we impose the constraint that the probabilities for each parent state \( t_i \) sum to 1:
\[
\sum_s \theta_s^i(t_i) = 1 \quad \text{for each } t_i , i 
\]
Using a Lagrange multiplier \( \lambda_{t_i}^i \) for each constraint, we construct the Lagrangian function:
\[
L = \sum_{n=1}^N \sum_{i=1}^K \log p\left(x_i^n \mid \operatorname{pa}(x_i^n)\right) + \sum_{i=1}^K \sum_{t_i} \lambda_{t_i}^i \left(1 - \sum_s \theta_s^i(t_i)\right)
\]

where:

\[
p\left(x_i^n \mid \operatorname{pa}(x_i^n)\right) = \sum_s \sum_{t_i}  \theta_s^i(t_i) ^{ \mathbb{I}\left[x_i^n = s\right] \mathbb{I}\left[\operatorname{pa}(x_i^n)\right] }
\]

Expanding the log-probability term in terms of \( \theta_s^i(t_i) \), we get:
\[
L = \sum_{n=1}^N \sum_{i=1}^K \sum_s \sum_{t_i}  \mathbb{I}\left[x_i^n = s\right] \mathbb{I}\left[\operatorname{pa}(x_i^n) = t_i\right] \log \theta_s^i(t_i) 
+ \sum_{i=1}^K \sum_{t_i} \lambda_{t_i}^i \left(1 - \sum_s \theta_s^i(t_i)\right)
\]
% where \( \mathbb{I}[\cdot] \) denotes the indicator function, which is 1 if the condition is true and 0 otherwise.

\textbf{Differentiation with Respect to \( \theta_s^i(t_i) \)}

To maximize \( L \) with respect to \( \theta_s^i(t_i) \), we take the partial derivative of \( L \) with respect to \( \theta_s^i(t_i) \) and set it to zero:
\[
\frac{\partial L}{\partial \theta_s^i(t_i)} = \sum_{n=1}^N \mathbb{I}\left[x_i^n = s\right] \mathbb{I}\left[\operatorname{pa}(x_i^n) = t_i\right] \frac{1}{\theta_s^i(t_i)} - \lambda_{t_i}^i = 0
\]
Rearranging, we obtain:
\[
\theta_s^i(t_i) = \frac{\sum_{n=1}^N \mathbb{I}\left[x_i^n = s\right] \mathbb{I}\left[\operatorname{pa}(x_i^n) = t_i\right]}{\lambda_{t_i}^i}
\]

\textbf{Solving for \( \lambda_{t_i}^i \) Using the Normalization Constraint}

We use the constraint \( \sum_s \theta_s^i(t_i) = 1 \) to determine \( \lambda_{t_i}^i \):
\[
\sum_s \theta_s^i(t_i) = \sum_s \frac{\sum_{n=1}^N \mathbb{I}\left[x_i^n = s\right] \mathbb{I}\left[\operatorname{pa}(x_i^n) = t_i\right]}{\lambda_{t_i}^i} = 1
\]
Therefore,
\[
\begin{aligned}    
\lambda_{t_i}^i 
&= \sum_s \sum_{n=1}^N \mathbb{I}\left[x_i^n = s\right] \mathbb{I}\left[\operatorname{pa}(x_i^n) = t_i\right]   \\
&= \sum_{n=1}^N  \sum_s \mathbb{I}\left[x_i^n = s\right] \mathbb{I}\left[\operatorname{pa}(x_i^n) = t_i\right]
\end{aligned}
\]

% Substituting \( \lambda_{t_i}^i \) Back

Substituting \( \lambda_{t_i}^i \) back into our expression for \( \theta_s^i(t_i) \),
 we have proven that the maximum likelihood estimate for \( \theta_s^i(t_i) \) is:
\[
\theta_s^i(t_i) = \frac{\sum_{n=1}^N \mathbb{I}\left[x_i^n = s\right] \mathbb{I}\left[\operatorname{pa}(x_i^n) = t_i\right]}{\sum_{n=1}^N \sum_s \mathbb{I}\left[x_i^n = s\right] \mathbb{I}\left[\operatorname{pa}(x_i^n) = t_i\right]}
\]

% ### Conclusion

% Thus, we have proven that the maximum likelihood estimate for \( \theta_s^i(t_i) \) is:
% \[
% \theta_s^i(t_i) = \frac{\sum_{n=1}^N \mathbb{I}\left[x_i^n = s\right] \mathbb{I}\left[\operatorname{pa}(x_i^n) = t_i\right]}{\sum_{n=1}^N \sum_s \mathbb{I}\left[x_i^n = s\right] \mathbb{I}\left[\operatorname{pa}(x_i^n) = t_i\right]}
% \]
% which is the required result.





\section*{Problem 4}

\subsection*{Part 1}

To count the number of possible Belief Networks with a given ancestral order, we analyze three cases:
\begin{enumerate}
    \item Networks with no restriction on the number of parents per node.
    \item Networks where each node has at most 2 parents.
    \item Networks where each node has at most \( k \) parents.
\end{enumerate}


\subsubsection*{Case 1: No Restriction on the Number of Parents per Node}

When there is no restriction on the number of parents, each node in a directed acyclic graph (DAG) can be connected to any subset of the previous nodes in the ancestral order. For \( n \) nodes, there are \( \binom{n}{2} \) possible directed edges, so the total number of unlabeled DAGs is:
\[
2^{\binom{n}{2}}
\]

\subsubsection*{Case 2: Restricting Each Node to At Most 2 Parents}

Let \( a_n \) denote the total number of unlabeled DAGs with \( n \) nodes, where each node has at most 2 parents. In this case, we aim to derive \( a_n \) recursively.

From observation, we have:
\[
a_1 = 1 \quad \text{and} \quad a_2 = 2
\]
since with one node there’s only one possible graph, and with two nodes there are two possible configurations (either connected or disconnected).

% #### Recursive Formula for the General Case

For \( n \geq 3 \), we consider three scenarios for the new \( n^{\text{th}} \) node, corresponding to it having 0, 1, or 2 parents from the previous \( n-1 \) nodes:
\begin{itemize}
    \item 0 parents: The new node is isolated, contributing \( a_{n-1} \) to the count.
    \item 1 parent: The new node is connected to one of the \( n-1 \) previous nodes, giving \( (n-1) \cdot a_{n-1} \).
    \item 2 parents: The new node is connected to two of the previous nodes, yielding \( \binom{n-1}{2} \cdot a_{n-1} \).
\end{itemize}

Thus, the recurrence relation is:
\[
a_n = a_{n-1} + (n-1) \cdot a_{n-1} + \binom{n-1}{2} \cdot a_{n-1}, \quad n \geq 3
\]

This recurrence has the closed-form solution:
\[
a_n = 2^{1 - n} \prod_{k=0}^{n-2} \left(\left(\frac{3}{2} + k\right)^2 + \frac{7}{4}\right)
\]

\subsubsection*{Case 3: Restricting Each Node to At Most \( k \) Parents}

For a general restriction of at most \( k \) parents per node, we define \( a_{n,k} \) as the total number of unlabeled DAGs with \( n \) nodes, where each node has up to \( k \) parents. We generalize the recurrence as follows:

% #### Base Cases

For \( i \leq k \), the number of possible DAGs with \( i \) nodes (where each node can connect freely to all previous nodes) is:
\[
a_{1,k} = 1
\]
\[
a_{i,k} = 2^{\binom{i}{2}} \quad   k \geq i \geq 2
\]


% #### Recursive Formula for \( n > k \)

For \( n > k \), we sum over the possible numbers of parents (from 0 up to \( k \)) for the new \( n^{\text{th}} \) node. Each case contributes a term \( \binom{n-1}{i} \cdot a_{n-1} \), where \( i \) is the number of parents the new node has:
\[
a_{n,k} = \sum_{i=0}^{k} \binom{n-1-k}{i} \cdot a_{n-1,k}, \quad n > k
\]

% ### Comparison of Counts

The table below compares the total number of possible DAGs with no parent restriction and the number with at most 2 parents per node:

\begin{table}[H]
\centering
\begin{tabular}{|c|c|c|c|}
\hline
\textbf{n} & \textbf{Total DAGs (at most 2 parents)} & \textbf{Possible Edges} & \textbf{Total DAGs (no restriction)} \\ \hline
1          & 1                     & 0                       & 1                  \\ \hline
2          & 2                     & 1                       & 2                  \\ \hline
3          & 8                     & 3                       & 8                  \\ \hline
4          & 56                    & 6                       & 64                 \\ \hline
5          & 616                   & 10                      & 1024               \\ \hline
6          & 9856                  & 15                      & 32768              \\ \hline
7          & 216832                & 21                      & 2097152            \\ \hline
8          & 6288128               & 28                      & 268435456          \\ \hline
9          & 232660736             & 36                      & 68719476736        \\ \hline
10         & 10702393856           & 45                      & 35184372088832     \\ \hline
\end{tabular}
\caption{Comparison of DAG counts with and without restrictions on parent nodes}
\end{table}

Therefore, the number of BN in  $N_a$ are $a_{8,2} = 6288128$

\subsection*{Part 2}

To calculate the computational time to find the optimal member of \( N_a \) where \( |N_a| = a_{8,2} \) using brute force (computing the BD score for each member of \( N_a \) individually), we would need \( |N_a| = 6,288,128 \) seconds (given that computing the BD score of any member of \( N_a \) takes 1 second).

However, by taking advantage of the decomposability of the BD score, we can reduce the computational time significantly. 
Since the BD score is decomposable, we can maximize the score for each node independently by maximizing the corresponding term for each node, represented by \( \prod_n p(v_k^n | \operatorname{pa}(v_k^n)) \). The BD score can be expressed as:
\[
p(\mathcal{D}) = \prod_k \prod_n p(v_k^n | \operatorname{pa}(v_k^n)) = \prod_k \prod_j \frac{Z(u'(v_k; j))}{Z(u(v_k; j))}
\]

Since it takes 1 second to compute the full BD score for any network and the BD score is composed of 8 independent terms (one for each node), we can assume that calculating each term individually takes approximately \( \frac{1}{8} \) seconds (the time heavily depends on the number of observations).

Next, we need to determine the number of different configurations for each node:

\begin{itemize}
    \item For the first node, there is only 1 configuration.
    \item For the second node, there are 2 configurations (either no connection or connected to the first node).
    \item For the \( n \)-th node, there are \( 1 + (n-1) + \binom{n-1}{2} \) configurations:
    (either no connection, connected to one of the previous nodes, or connected to two of the previous nodes).
\end{itemize}

We can summarize the number of configurations for each node \( n \) as follows:

\[
b_n = n + \frac{(n-1)(n-2)}{2} \quad n \geq 1
\]

Thus, for each node, we need to find the configuration that maximizes its corresponding term out of the \( b_n \) configurations. Each configuration requires \( \frac{1}{8} \) seconds to evaluate.

The total time to find the optimal member of \( N_a \) is therefore:

\[
\sum_{n=1}^{8} \frac{1}{8} b_n = \frac{1}{8} \sum_{n=1}^{8} \left( n + \frac{(n-1)(n-2)}{2} \right) = \frac{92}{8} = 11.5 \text{ seconds}
\]

This calculation demonstrates that, by leveraging decomposability, we reduce the time from millions of seconds to only 11.5 seconds.

\subsection*{Part 3}

First, it is important to note that \( |N| < 8! |N_a| \). For a general case without the restriction of at most 2 parents per node, the solution is not \( n! \cdot 2^{\binom{n-1}{2}} \); instead, it can be found \href{https://en.wikipedia.org/wiki/Directed_acyclic_graph#Combinatorial_enumeration}{here}.

One way to estimate the total time required to find the optimal member of \( N \) is to check every possible ancestral order \( a \) (of which there are \( 8! \)). Using this approach, the total time would be:
\[
8! \cdot 11.5 = 463680 \text{ seconds.}
\]

It would be interesting to investigate whether a more efficient approach exists to speed up the above calculation. One potential (but untested) idea could involve optimizing both the order and structure simultaneously. By carefully considering the order of operations, this might lead to:
\[
\sum_{n=1}^{8} \frac{1}{8} b_n \cdot (8 - n) = \frac{1}{8} \sum_{n=1}^{8} \left( (8 - n) \left( n + \frac{(n-1)(n-2)}{2} \right) \right) = \frac{246}{8} = 30.75 \text{ seconds.}
\]

This approach is speculative, and further analysis or testing would be needed to determine whether it is correct and effective.

\end{document}
